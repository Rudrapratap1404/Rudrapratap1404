\documentclass[a4paper,11pt,twocolumn]{article}
\usepackage[a4paper,left=1.5cm,right=1cm,top=2cm,bottom=2cm]{geometry}
\usepackage{setspace}
\usepackage{gensymb}
\usepackage{caption}
\usepackage{graphicx}
\usepackage{tabularx}
\usepackage{lmodern}
\usepackage{watermark} 
\usepackage{lipsum}
\usepackage{xcolor}
\usepackage{listings}
\usepackage{graphicx}
\usepackage{enumitem}
\usepackage{mathtools}
\usepackage{titlesec}
\usepackage[utf8]{inputenc}
\usepackage{fontenc}
\usepackage{karnaugh-map}
\usepackage{tikz}
\usepackage{harvard}
\usepackage{amsfonts}
\graphicspath{{/storage/emulated/0/Download/FWC/Latex/figs}}
\usepackage[colorlinks,linkcolor={black},citecolor={blue!80!black},urlcolor={blue!80!black}]{hyperref}
\title{\textbf{\textsc{VERIFICATION OF BOOLEAN IDENTITIES}}}
\author{\textbf{\textit{\teflipflopxtbf{Rudra Pratap Singh}}}}
\begin{document}

\date{}
\maketitle
\tableofcontents



\section{PROBLEM}
\textbf{(GATE CS-2008)}
\textbf{Q.5} In the Karnaugh map shown below, X denotes a don't care term. What is the minimal form of the function represented by the Karnaugh map?


\begin{karnaugh-map}[4][4][1][][]
    \maxterms{3,5,6,7,11,13,14,15}
    \minterms{0,1,2,8,9}
    \terms{4,10,12}{X} 
    % note: posistion for start of \draw is (0, Y) where Y is
    % the Y size(number of cells high) in this case Y=2
    \draw[color=black, ultra thin] (0, 4) --
    node [pos=0.7, above right, anchor=south west] {$ba$} % Y label
    node [pos=0.7, below left, anchor=north east] {$cd$} % X label
    ++(135:1);
        
    \end{karnaugh-map}
\begin{enumerate}[label=(\Alph*)]
	\item $ b'd'+a'd' $
	\item $ a'b'+b'd'+a'bd' $
	\item $ b'd'+a'bd' $
	\item $ a'b'+b'd'+a'd' $

 \begin{karnaugh-map}[4][4][1][Table 1][]
    \maxterms{3,5,6,7,11,13,14,15}
    \minterms{0,1,2,8,9}
    \terms{4,10,12}{X} 
    \implicantedge{0}{1}{8}{9}
    \tikzset{
        greenimplicant/.style={
            pattern color=green,
            pattern=north west lines,
        },
    }

    % Mark implicants with a green pattern
    \implicantedge{0}{1}{8}{9}
    \color{green}
    % note: posistion for start of \draw is (0, Y) where Y is
    % the Y size(number of cells high) in this case Y=2
    \draw[color=black, ultra thin] (0, 4) --
    node [pos=0.7, above right, anchor=south west] {$ba$} % Y label
    node [pos=0.7, below left, anchor=north east] {$cd$} % X label
    ++(135:1);
        
    \end{karnaugh-map}
 
\begin{karnaugh-map}[4][4][1][Table 2][]
    \maxterms{3,5,6,7,11,13,14,15}
    \minterms{0,1,2,8,9}
    \terms{4,10,12}{X} 
     \implicantcorner 
     

    % note: posistion for start of \draw is (0, Y) where Y is
    % the Y size(number of cells high) in this case Y=2
    \draw[color=black, ultra thin] (0, 4) --
    node [pos=0.7, above right, anchor=south west] {$ba$} % Y label
    node [pos=0.7, below left, anchor=north east] {$cd$} % X label
    ++(135:1);
\end{karnaugh-map}

\bigskip
                        
One group is $\{0000, 0001, 1000, 1001\}$ (Table 1) which gives $a'd'$. \\
Another group consists of $\{0000, 0010, 1000, 1010\}$ (Table 2) which gives $b'd'$. \\
Hence, the correct option is \textbf{a)}.




\end{enumerate}
\bigskip

\section{COMPONENTS}
	\begin{tabularx}{0.45\textwidth} {  
  | >{\centering\arraybackslash}X  
  | >{\centering\arraybackslash}X  
  | >{\centering\arraybackslash}X | } 
\hline 
\textbf{Component} &  \textbf{Value} & \textbf{Quantity}\\ 
\hline 
Arduino & UNO & 1 \\   
\hline 
Bread board & - & 1 \\ 
\hline
Jumper wires & M-M & 10 \\ 
\hline 
LED & - & 1\\ 

\hline 
\end{tabularx}
 
\bigskip 

\section{TRUTH TABLE}
The Truth Table for the identity is as follows:
\begin{enumerate}[label=\textbf{(\Alph*})]
	\item \textbf{ $ Y=b'd'+a'd' $} \\
\bigskip
\begin{table}[ht!]
	\centering
\begin{tabular}{ |c |c |c |c |c |c |} 
\hline 
\newline 
	\textbf{a} & \textbf{b} & \textbf{d} & \textbf{Y} \\ 
\hline  
	0 & 0 & 0 &1 \\   
	0 & 0 & 1 &0 \\  
	0 & 1 & 0 &1 \\  
	0 & 1 & 1 &0 \\  
	1 & 0 & 0 &1 \\  
	1 & 0 & 1 &0 \\  
	1 & 1 & 0 &0 \\  
	1 & 1 & 1 &0 \\  
\hline 
\end{tabular}
	\caption{}
\end{table}
\bigskip
\bigskip


\end{enumerate}
\section{ARDUINO CONNECTIONS}

1) The connections taken from Arduino as Input and Output is as follows:
\begin{table}[ht!] 
    \centering 
    \begin{tabular}{|c|c|c|c|c|c|c|c|} 
    \hline 
       Input & $a$&$b$&$d$&$Y$\\ 
       \hline 
    Arduino & 3&4&5&6\\ 
    \hline 
    \end{tabular} 
    \caption{} 
\end{table} 
\\
2) The  input \textbf{a,b,c} here are connected to Arduino D3,D4,D5 pins.\\
3) The  output \textbf{Y} here are connected to Arduino D6 pins.\\
4) The values for these inputs are conncted either to GND or 5V according to the truth table.\\
\section{CODE}
\paragraph{}
	The arduino code can be downloaded from the below link.
\begin{center} 
\fbox{\parbox{8.5cm}{\url{https://github.com/Rudrapratap1404/Rudrapratap1404/blob/main/ide/code/src/GATE2008.cpp}}} 
\end{center}


\end{document}